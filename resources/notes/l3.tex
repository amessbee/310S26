\documentclass[11pt]{article}
\usepackage{amsmath, amssymb}
\usepackage{geometry}
\usepackage{booktabs}
\usepackage{hyperref}
\usepackage{listings}
\usepackage{xcolor}

\geometry{margin=1in}

\title{Notes (Lecture 3)}
\author{}
\date{}

\lstset{
  basicstyle=\ttfamily\small,
  frame=single,
  breaklines=true
}

\begin{document}

\maketitle

\section{The Selection Problem \& Prune-and-Search}

\subsection{Problem Statement}

You are given an array $A$ of $n$ \textbf{distinct} elements. 

Your goal is to find the \textbf{$k^{th}$ smallest element} efficiently.

\textbf{Example:}

If  
\[
A = [7, 2, 1, 6, 8, 5, 3, 4]
\]
and $k = 3$, then the answer is $\mathbf{3}$.

\subsection{Looking at the Naive Approach}

Sort the array and return the $k^{th}$ element.

\begin{itemize}
  \item Sorting takes $O(n \log n)$.
  \item But we do not need the full order, only one element.
\end{itemize}

So we want something faster.

\subsection{Idea: Prune and Search}

Instead of sorting everything, we:

\begin{quote}
Guess, partition, and discard the part that cannot contain the answer.
\end{quote}

This is known as the \textbf{Prune-and-Search paradigm}.  
Each step removes many unnecessary elements.

\section{General Select Algorithm (Prune \& Search)}

\subsection{Step 1: Pick a Guess $g$}

Choose an element from the array as a \textbf{pivot}.

\begin{itemize}
  \item Deterministic: Median of Medians.
  \item Randomized: Pick uniformly at random.
\end{itemize}

\subsection{Step 2: Partition}

Split the array into:

\[
L = \{x \mid x < g\}, \quad R = \{x \mid x > g\}
\]

Everything smaller goes left and everything larger goes right.

This takes:
\[
O(n)
\]

\subsection{Step 3: Decide}

Let $|L|$ be the size of $L$.

\begin{itemize}
  \item If $|L| = k-1$, then $g$ is the answer.
  \item If $|L| \ge k$, search in $L$ for the $k^{th}$ smallest.
  \item If $|L| < k-1$, search in $R$ for the $(k-|L|-1)^{th}$ smallest.
\end{itemize}

Each round throws away part of the array.

\subsection{Pseudocode}

\begin{lstlisting}
Select(A, k):
    choose pivot g from A
    partition A into L and R

    if |L| == k-1:
        return g
    else if |L| >= k:
        return Select(L, k)
    else:
        return Select(R, k - |L| - 1)
\end{lstlisting}

\section{Dry Run Example}

Given:
\[
A = [7, 2, 1, 6, 8, 5, 3, 4], \quad k = 3
\]

\subsection*{Step 1}

Pick $g = 5$.

\subsection*{Step 2}

\[
L = [2,1,3,4], \quad R = [7,6,8]
\]
\[
|L| = 4
\]

\subsection*{Step 3}

Since $|L| \ge k$, we recurse:

\[
Select([2,1,3,4], 3)
\]

\subsection*{Next Round}

Pick $g = 2$.

\[
L = [1], \quad R = [3,4]
\]
\[
|L| = 1
\]

Since $|L| < k-1$, compute:

\[
k' = 3 - 1 - 1 = 1
\]

Recurse:

\[
Select([3,4], 1)
\]

\subsection*{Next Round}

Pick $g = 3$.

\[
L = []
\]
\[
|L| = 0 = k-1
\]

Thus the answer is:

\[
\boxed{3}
\]

Each step shrinks the problem size.

\section{Deterministic vs Randomized Selection}

\begin{center}
\begin{tabular}{lcc}
\toprule
Feature & Deterministic & Randomized \\
\midrule
Pivot Choice & Median of Medians & Random \\
Worst Case & $O(n)$ & $O(n^2)$ \\
Expected & $O(n)$ & $\Theta(n)$ \\
\bottomrule
\end{tabular}
\end{center}

\subsection{Deterministic (Median of Medians)}

Median of Medians guarantees a good pivot so that a constant fraction is discarded each step.

The standard recurrence is:

\[
T(n) \le T(n/5) + T(7n/10) + cn
\]

which solves to:

\[
T(n) = O(n).
\]


\section{Why Randomize?}

Even though the worst case is $O(n^2)$, it is extremely unlikely. The probability of picking a bad pivot at every step is roughly $(1/n)^n$, which is effectively zero for large $n$. Hence, although the worst case exists, the expected running time remains linear in practice.

\end{document}
