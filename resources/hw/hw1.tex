\documentclass[9pt]{book}
\setlength{\oddsidemargin}{-.5 in}
\setlength{\evensidemargin}{-.5 in}
% \setlength{\evensidemargin}{.0 in}
\addtolength{\topmargin}{-1in}
\setlength{\textwidth}{7.5in}
\setlength{\textheight}{9in}
\usepackage{enumitem}
\usepackage{xcolor}

%\usepackage[utf8x]{inputenc}
\usepackage{amsmath, amssymb}
%\usepackage{algorithm}
%\usepackage{algorithmic}
%\usepackage[normalem]{ulem}
%\usepackage[switch, mathlines, displaymath]{lineno}
%\linenumbers
\usepackage{algorithm}
\usepackage{url, color, epsfig, amsmath, amssymb}
\usepackage{graphicx}
\usepackage{amsmath, amsthm, amssymb, nicefrac}
%\usepackage{algorithmic}
\usepackage{algpseudocode}

% \usepackage{times}
\usepackage{algorithm}
\usepackage{url, color, epsfig, amsmath, amsthm, amssymb}
\usepackage{amsthm}
\usepackage{graphicx}
\usepackage{mathtools}
\DeclarePairedDelimiter{\ceil}{\lceil}{\rceil}
\DeclarePairedDelimiter{\floor}{\lfloor}{\rfloor}
%
% this command enables to remove a whole part of the text
% from the printout
% to use it just enter
% \remove{
% before the text to be excluded and
% }
% after the text
\newcommand{\remove}[1]{}

%
% The following macros are used to generate nice code for programs.
% See example on how to use it below
%

%%%%%%%%%%%%%%%%%%%%% program macros %%%%%%%%%%%%%%%%%

\newcommand{\Do}{{\small\bf do}\ }
\newcommand{\Proc}[1]{#1\+}
\newcommand{\Returns}{{\small\bf returns}}
\newcommand{\Procbegin}{{\small\bf begin}}
\newcommand{\Then}{{\small\bf then}\ \=\+}
\newcommand{\Elseif}{\<{\small\bf elseif}\ }
\newcommand{\Endif}{\<{\small\bf end\ if\ }\-\-}
\newcommand{\Endproc}[1]{{\small\bf end} #1\-}
\newcommand{\Endfor}{{\small\bf end\ for}\ \-}
\newcommand{\Endloop}{{\bf end\ loop}\ \-}
\newenvironment{program}{
	\begin{minipage}{\textwidth}
		\begin{tabbing}
			\ \ \ \ \=\kill
		}{
	\end{tabbing}
\end{minipage}
}

% a blank line
\def\blankline{\hbox{}}


\newcommand{\lecture}[5]{
	\pagestyle{headings}
	\thispagestyle{plain}
	\newpage
	\setcounter{chapter}{#1}
	\setcounter{page}{#2}
	%  \set\thechapter{#3}
	\noindent
	\begin{center}
		\framebox{
			\vbox{
				\hbox to 6.28in { {\bf Algorithms 
						\hfill Spring Semester, 2026} }
				\vspace{4mm}
				\hbox to 6.28in { {\Large \hfill Homework 1 \hfill} }
				\vspace{2mm}
				\hbox to 6.28in { {\it Professor: #4 \hfill Deadline: Feb 02, 2026.} }
			}
		}


	\end{center}
	% \markboth{Lecture #1: #3}{Lecture #1: #3}
	\vspace*{4mm}
}


\newcommand{\topic}[2]{\section{#1} \index{#2} \markright{#1}}
\newcommand{\subtopic}[2]{\subsection{#1} \index{#2}}
\newcommand{\subsubtopic}[2]{\subsubsection{#1} \index{#2}}


\renewcommand{\cite}[1]{[#1]}
\renewcommand{\thefootnote}{\fnsymbol{footnote}}
%
% These are just to make things a little easier:
%
\newcommand{\bi}{\begin{itemize}}
	\newcommand{\ei}{\end{itemize}}
\newcommand{\be}{\begin{enumerate}}
	\newcommand{\ee}{\end{enumerate}}
\newcommand{\blank}{\vspace{1ex}}   % generates a blank line in the output

\newtheorem{theorem}{Theorem}[chapter]
\newtheorem{lemma}[theorem]{Lemma}
\newtheorem{claim}[theorem]{Claim}
\newtheorem{digress}[theorem]{Digression}
\newtheorem{corollary}[theorem]{Corollary}

\newenvironment{dfn}{{\vspace*{1ex} \noindent \bf Definition }}{\vspace*{1ex}}
\newcommand{\bigdef}[2]{\index{#1}\begin{dfn} {\rm #2} \end{dfn}}
\newcommand{\smalldef}[1]{\index{#1} {\em #1}}

\begin{document}
	%\lecture{**LECTURE-NUMBER**}{**1ST-PAGE**}{**DATE**}{**LECTURER**}{**SCRIBE**}
	\lecture{0}{1}{\today}{Dr. Mudassir Shabbir}{180 mins.}
%Write up convincing answers to the questions below. The solution to every problem should appear on a new page. Every solution must be concise and should not take more than an A4 page. \textbf{Ten points will be deducted from total for every extra page taken.} Please upload a single image or PDF to classroom for each problem. \textbf{ Concentrate on coming up with concise, clean, and crisp write-up. You may receive extra credit for exceptionally well-written arguments.} You are expected to solve first four problems. The rest of the questions are {\bf bonus problems (marked with *)}, that you may try if you have any time left. Contact your TA/instructor in case of an issue.
%Please solve the following problem, and submit a concise answer before midnight on Tuesday. The problem weighs $20$ absolute points towards your overall semester score.
The solution to every problem should appear on a new page. 
Every solution must be concise and should not take more than an A4 page.
\textbf{ You may receive extra credit for exceptionally well-written arguments.} 
% Provide the most efficient algorithm you can come up with and always provide its time complexity. 
Contact your TA/instructor in case of an issue.

%General instructions:
%\begin{itemize}
	%TODO Change here
%	\item The pages in this book are numbered from $1$ to $8$ - please check before you proceed.
%	\item The exam is open-book, open-notes. However you are not allowed to use calculators, cell phones, tablets, computers and other electronic devices. You are NOT allowed to share your resources with anyone else either.
%	\item Read the problem statements carefully - some hints/definitions are also provided in the footnotes. Always provide reason for your answer even
%	when problem doesn't explicitly ask for it.
%	\item Each problem is marked with the points it carries - the exam carries $75$ points. Since you have $90$	minutes, you can spare - on average - $72$ seconds on each point.
	%\item Unless otherwise advised, every graph is {\em a simple graph} i.e. undirected unweighted with no loops and no
	%multiple edges.
%	\item All problems carry equal weight
%	\item Rule 25 applies to all problems in this exam. To use it, you have to write `IDK' against any problem you want to receive $25 \%$ credit for.
%	\item Bonus problems are marked with a double asterisk(**); they are not mandatory and you can't use Rule 25 for them. You are encouraged to attempt them once done with the rest of the problems.
%		\item Provide your name, ID(or roll no), and the location of your current seat in the box below. Your location 		is in the form of $X\times Y$ where $X$ is the row number and $Y$ is the column number.
%\end{itemize}
% **** YOUR NOTES GO HERE
%\vspace*{30pt}

 %     \framebox{
  %    	\vbox{
  %    		\hbox to 6.28in { {\bf Name: \hfill ID:\hfill} }
  %    		\vspace{4mm}
  %    		\hbox to 6.28in { {\bf Seat Location: \hfill } }
  %    	}
  %    }
%\newpage
%    Some general latex examples and examples making use of the
% macros follow.  You may want to latex this file and print out the result
% to get an idea of how things look. (You will need to latex the file twice
% to get the cross-references right.)
%    Even if you already know latex, you should take a few minutes to
% look over these examples, so that you understand the general conventions
% we're using.


\section*{Asymptotic Analysis}
\subsection*{Question 1: \hfill[5]}
Prove or Disprove the following statements\footnote{Note that we say $f(n) = o(g(n))$ if
\[
\lim_{n \to \infty} \frac{f(n)}{g(n)} = 0.
\]
Equivalently, for every constant $c > 0$, there exists $n_0$ such that
$f(n) \le c\, g(n)$ for all $n \ge n_0$. Informally: \textit{$f$ grows strictly slower than $g$}.}:
\begin{enumerate}
\item $f(n) + g(n) = \Theta(\min(f(n),g(n)))$
\item $f(n) = O(g(n)) \rightarrow \log(f(n)) = O(\log(g(n)))$
\item $f(n) = O(g(n)) \rightarrow g(n) = \Omega(f(n))$
\item $f(n) = o(g(n)) \rightarrow \log(f(n)) = o(\log(g(n)))$


\end{enumerate}

\subsection*{Question 2: \hfill[5]}
Choose whether $f(n)$ belongs to one or more of the following: {$O(g(n)), \Theta(g(n)),\Omega(g(n))$} \\ %\quad o(g(n)) \quad \omega(g(n))$} \\

\renewcommand{\arraystretch}{2} 
\begin{table}[H]
\centering
\begin{tabular}{|c|c|c|c|c|c|}
\hline
\textbf{\#} & $f(n)$ & $g(n)$ & $O$ & $\Omega$ & $\Theta$ \\ \hline
1 & $n^{-2}$ & $\frac{1}{\sqrt[2]{n}}$ & \quad & \quad & \quad \\ \hline
2 & $2^n$ & $n^{n/2}$ & \quad & \quad & \quad \\ \hline
3 & $\sqrt[3]{n^n}$ & $n^{n/3}$ & \quad & \quad & \quad \\ \hline
4 & $\log(2n^2)$ & $\log(4n^3) - \log(2n)$ & \quad & \quad & \quad \\ \hline
5 & $n^{-3}$ & $2^{50}$ & \quad & \quad & \quad \\ \hline
6 & $\log(n)$ & $\sqrt{n}$ & \quad & \quad & \quad \\ \hline
7 & $n!$ & $n^n$ & \quad & \quad & \quad \\ \hline
8 & $n\log(n!)$ & $\log(n!)$ & \quad & \quad & \quad \\ \hline
\end{tabular}
\end{table}

\newpage 
\section*{Order Statistics}
The selection problem is to find the $ith$ order statistic of a list. In class, we studied the deterministic linear-time selection algorithm SELECT (also known as Median of Medians).

\subsection*{Question 3: \hfill[5]}
Show how to use SELECT as a subroutine to make Quicksort run in $O(n\log(n))$ time in the worst case, assuming that all elements are distinct. \\
(* Refer to relevant parts of CLRS book for a description of Quicksort).

\subsection*{Question 4: \hfill[5]}
The SELECT algorithm divides the list into subgroups of size $c$, where $c$ is typically set to $5$. What effect would there be on the algorithm's worst-case run-time if is chosen to be a) less than $5$, b) more than $5$, and c) if $c$ becomes a function of $n$, i.e., $\log n, \sqrt{n}, n/2, n$?

\subsection*{Question 5: \hfill [5]}
Design a linear time algorithm to find all of the $\log n$ smallest integers, {\color{blue} in sorted order}, using the SELECT problem as a subroutine. For example, if we have a list of $n=2^{100}$ integers, then we are interested in the $100$ smallest integers in the given list.

\section*{Randomized Algorithms}
RANDOMIZED-SELECT is a selection algorithm where pivots are chosen {\em uniformly} at random, unlike the Median of Medians problem which is a deterministic-linear algorithm.
\subsection*{Question 6: \hfill[5]}
Show an example where RANDOMIZED-SELECT has a worst-case run-time of $O(n^2)$

\subsection*{Question 7: \hfill[5]}
Compute the probability that a randomly chosen pivot splits the array into two parts each of size at least $\frac{3n}{10}$. Use this to show that the expected run-time of RANDOMIZED-SELECT is $\Theta(n)$.

\subsection*{Question 8: \hfill[5]}
Suppose we want RANDOMIZED-SELECT to succeed in finding the $k$-th smallest element \emph{without hitting a worst-case path}. If the probability of a good split is $p$, what is the probability that after $t$ independent pivot choices, we always hit good splits?
\end{document}
