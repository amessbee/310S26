\documentclass[9pt]{book}
\setlength{\oddsidemargin}{-.5 in}
\setlength{\evensidemargin}{-.5 in}
% \setlength{\evensidemargin}{.0 in}
\addtolength{\topmargin}{-1in}
\setlength{\textwidth}{7.5in}
\setlength{\textheight}{9in}
\usepackage{enumitem}
\usepackage{xcolor}

%\usepackage[utf8x]{inputenc}
\usepackage{amsmath, amssymb}
%\usepackage{algorithm}
%\usepackage{algorithmic}
%\usepackage[normalem]{ulem}
%\usepackage[switch, mathlines, displaymath]{lineno}
%\linenumbers
\usepackage{algorithm}
\usepackage{url, color, epsfig, amsmath, amssymb}
\usepackage{graphicx}
\usepackage{amsmath, amsthm, amssymb, nicefrac}
%\usepackage{algorithmic}
\usepackage{algpseudocode}

% \usepackage{times}
\usepackage{algorithm}
\usepackage{url, color, epsfig, amsmath, amsthm, amssymb}
\usepackage{amsthm}
\usepackage{graphicx}
\usepackage{mathtools}
\DeclarePairedDelimiter{\ceil}{\lceil}{\rceil}
\DeclarePairedDelimiter{\floor}{\lfloor}{\rfloor}
%
% this command enables to remove a whole part of the text
% from the printout
% to use it just enter
% \remove{
% before the text to be excluded and
% }
% after the text
\newcommand{\remove}[1]{}

%
% The following macros are used to generate nice code for programs.
% See example on how to use it below
%

%%%%%%%%%%%%%%%%%%%%% program macros %%%%%%%%%%%%%%%%%

\newcommand{\Do}{{\small\bf do}\ }
\newcommand{\Proc}[1]{#1\+}
\newcommand{\Returns}{{\small\bf returns}}
\newcommand{\Procbegin}{{\small\bf begin}}
\newcommand{\Then}{{\small\bf then}\ \=\+}
\newcommand{\Elseif}{\<{\small\bf elseif}\ }
\newcommand{\Endif}{\<{\small\bf end\ if\ }\-\-}
\newcommand{\Endproc}[1]{{\small\bf end} #1\-}
\newcommand{\Endfor}{{\small\bf end\ for}\ \-}
\newcommand{\Endloop}{{\bf end\ loop}\ \-}
\newenvironment{program}{
	\begin{minipage}{\textwidth}
		\begin{tabbing}
			\ \ \ \ \=\kill
		}{
	\end{tabbing}
\end{minipage}
}

% a blank line
\def\blankline{\hbox{}}

%%%%%%%%%%%%%%%%%%%%% End of PROGRAM macros %%%%%%%%%%%%%%%%%


%
% The following macro is used to generate the header.
%
%

\newcommand{\lecture}[5]{
	\pagestyle{headings}
	\thispagestyle{plain}
	\newpage
	\setcounter{chapter}{#1}
	\setcounter{page}{#2}
	%  \set\thechapter{#3}
	\noindent
	\begin{center}
		\framebox{
			\vbox{
				\hbox to 6.28in { {\bf Algorithms 
						\hfill Spring Semester, 2026} }
				\vspace{4mm}
				\hbox to 6.28in { {\Large \hfill Homework 1 \hfill} }
				\vspace{2mm}
				\hbox to 6.28in { {\it Professor: #4 \hfill Deadline: Feb 02, 2026.} }
			}
		}


	\end{center}
	% \markboth{Lecture #1: #3}{Lecture #1: #3}
	\vspace*{4mm}
}

%
% Use these macros for organizing sections of your notes.
% Each command takes two arguments: (1) the title of the section and and
% (2) a keyword for that section to appear in the index.  (See examples.)
% Please don't use \section, \subsection, and \subsubsection directly!
%

\newcommand{\topic}[2]{\section{#1} \index{#2} \markright{#1}}
\newcommand{\subtopic}[2]{\subsection{#1} \index{#2}}
\newcommand{\subsubtopic}[2]{\subsubsection{#1} \index{#2}}

%
% Convention for citations is first author's last name followed by other
% authors' last initials, followed by the year.  For example, to cite the
% seventh entry in the course bibliography, you would type: \cite{BurnsL80}
% (To avoid bibliography problems, for now we redefine the \cite command.)
%

\renewcommand{\cite}[1]{[#1]}
\renewcommand{\thefootnote}{\fnsymbol{footnote}}
%
% These are just to make things a little easier:
%
\newcommand{\bi}{\begin{itemize}}
	\newcommand{\ei}{\end{itemize}}
\newcommand{\be}{\begin{enumerate}}
	\newcommand{\ee}{\end{enumerate}}
\newcommand{\blank}{\vspace{1ex}}   % generates a blank line in the output

%
% Use these for theorems, lemmas, proofs, etc.
%
\newtheorem{theorem}{Theorem}[chapter]
\newtheorem{lemma}[theorem]{Lemma}
\newtheorem{claim}[theorem]{Claim}
\newtheorem{digress}[theorem]{Digression}
\newtheorem{corollary}[theorem]{Corollary}
%\newcommand{\qed}{\hfill $\Box$}
%\newenvironment{proof}{\par{\bf Proof:}}{\qed \par}
%\newenvironment{proof}{{\em Proof:}}{\hfill\rule{2mm}{2mm}}

%
% Use the following for definitions.
% \bigdef is for definitions to be set off by themselves; \smalldef is for
% definitions given in the middle of a paragraph.
%
\newenvironment{dfn}{{\vspace*{1ex} \noindent \bf Definition }}{\vspace*{1ex}}
\newcommand{\bigdef}[2]{\index{#1}\begin{dfn} {\rm #2} \end{dfn}}
\newcommand{\smalldef}[1]{\index{#1} {\em #1}}

\begin{document}
	%\lecture{**LECTURE-NUMBER**}{**1ST-PAGE**}{**DATE**}{**LECTURER**}{**SCRIBE**}
	\lecture{0}{1}{\today}{Dr. Mudassir Shabbir}{180 mins.}
%Write up convincing answers to the questions below. The solution to every problem should appear on a new page. Every solution must be concise and should not take more than an A4 page. \textbf{Ten points will be deducted from total for every extra page taken.} Please upload a single image or PDF to classroom for each problem. \textbf{ Concentrate on coming up with concise, clean, and crisp write-up. You may receive extra credit for exceptionally well-written arguments.} You are expected to solve first four problems. The rest of the questions are {\bf bonus problems (marked with *)}, that you may try if you have any time left. Contact your TA/instructor in case of an issue.
%Please solve the following problem, and submit a concise answer before midnight on Tuesday. The problem weighs $20$ absolute points towards your overall semester score.
The solution to every problem should appear on a new page. 
Every solution must be concise and should not take more than an A4 page.
\textbf{ You may receive extra credit for exceptionally well-written arguments.} 
% Provide the most efficient algorithm you can come up with and always provide its time complexity. 
Contact your TA/instructor in case of an issue.

%General instructions:
%\begin{itemize}
	%TODO Change here
%	\item The pages in this book are numbered from $1$ to $8$ - please check before you proceed.
%	\item The exam is open-book, open-notes. However you are not allowed to use calculators, cell phones, tablets, computers and other electronic devices. You are NOT allowed to share your resources with anyone else either.
%	\item Read the problem statements carefully - some hints/definitions are also provided in the footnotes. Always provide reason for your answer even
%	when problem doesn't explicitly ask for it.
%	\item Each problem is marked with the points it carries - the exam carries $75$ points. Since you have $90$	minutes, you can spare - on average - $72$ seconds on each point.
	%\item Unless otherwise advised, every graph is {\em a simple graph} i.e. undirected unweighted with no loops and no
	%multiple edges.
%	\item All problems carry equal weight
%	\item Rule 25 applies to all problems in this exam. To use it, you have to write `IDK' against any problem you want to receive $25 \%$ credit for.
%	\item Bonus problems are marked with a double asterisk(**); they are not mandatory and you can't use Rule 25 for them. You are encouraged to attempt them once done with the rest of the problems.
%		\item Provide your name, ID(or roll no), and the location of your current seat in the box below. Your location 		is in the form of $X\times Y$ where $X$ is the row number and $Y$ is the column number.
%\end{itemize}
% **** YOUR NOTES GO HERE
%\vspace*{30pt}

 %     \framebox{
  %    	\vbox{
  %    		\hbox to 6.28in { {\bf Name: \hfill ID:\hfill} }
  %    		\vspace{4mm}
  %    		\hbox to 6.28in { {\bf Seat Location: \hfill } }
  %    	}
  %    }
%\newpage
%    Some general latex examples and examples making use of the
% macros follow.  You may want to latex this file and print out the result
% to get an idea of how things look. (You will need to latex the file twice
% to get the cross-references right.)
%    Even if you already know latex, you should take a few minutes to
% look over these examples, so that you understand the general conventions
% we're using.


\section*{Asymptotic Analysis}
\subsection*{Question 1: \hfill[5]}
Prove or Disprove the following statements\footnote{Note that we say $f(n) = o(g(n))$ if
\[
\lim_{n \to \infty} \frac{f(n)}{g(n)} = 0.
\]
Equivalently, for every constant $c > 0$, there exists $n_0$ such that
$f(n) \le c\, g(n)$ for all $n \ge n_0$. Informally: \textit{$f$ grows strictly slower than $g$}.}:
\begin{enumerate}
\item $f(n) + g(n) = \Theta(\min(f(n),g(n)))$
\item $f(n) = O(g(n)) \rightarrow \log(f(n)) = O(\log(g(n)))$
\item $f(n) = O(g(n)) \rightarrow g(n) = \Omega(f(n))$
\item $f(n) = o(g(n)) \rightarrow \log(f(n)) = o(\log(g(n)))$


\end{enumerate}



% \subsection*{Question 2: \hfill[5]}
% Prove the following relationships by giving the appropriate $c$ and $n_0$, or disprove:
% \begin{enumerate}
% \item $3\log(n) \in O(\log(n^2))$
% \item $\log^2n \in O(\log(n^2))$
% \item $\frac{1}{\sqrt[3][n]} \in \Omega(\frac{1}{\sqrt[2][n]})$
% \item $2^n \in \Theta(2^{2n})$
% \item $5^n \in \Omega(5^{n+50})$
% \end{enumerate}

\subsection*{Question 2: \hfill[5]}
Choose whether $f(n)$ belongs to one or more of the following: {$O(g(n)), \Theta(g(n)),\Omega(g(n))$} \\ %\quad o(g(n)) \quad \omega(g(n))$} \\

% \textcolor{red}{
% \begin{itemize}
% \item a function $f(n)$ belongs to little o $o(g(n))$ if $f(n)$ is strictly upper bounded by $g(n)$, never equal. For example, $n \in o(n^2)$.
% \item a function $f(n)$ belongs to little omega $\omega(g(n))$ if $f(n)$ is strictly lower bounded by $g(n)$, never equal. For example, $n^2 \in \omega(n)$.
% \end{itemize}
% }
\renewcommand{\arraystretch}{2} 
\begin{table}[H]
\centering
\begin{tabular}{|c|c|c|c|c|c|}
\hline
\textbf{\#} & $f(n)$ & $g(n)$ & $O$ & $\Omega$ & $\Theta$ \\ \hline
1 & $n^{-2}$ & $\frac{1}{\sqrt[2]{n}}$ & \quad & \quad & \quad \\ \hline
2 & $2^n$ & $n^{n/2}$ & \quad & \quad & \quad \\ \hline
3 & $\sqrt[3]{n^n}$ & $n^{n/3}$ & \quad & \quad & \quad \\ \hline
4 & $\log(2n^2)$ & $\log(4n^3) - \log(2n)$ & \quad & \quad & \quad \\ \hline
5 & $n^{-3}$ & $2^{50}$ & \quad & \quad & \quad \\ \hline
6 & $\log(n)$ & $\sqrt{n}$ & \quad & \quad & \quad \\ \hline
7 & $n!$ & $n^n$ & \quad & \quad & \quad \\ \hline
8 & $n\log(n!)$ & $\log(n!)$ & \quad & \quad & \quad \\ \hline
\end{tabular}
\end{table}


% \subsection*{Question 4: \hfill[10]}
% Order the following run-time complexities from slowest to fastest growth. Any functions with the same run-time complexity should be grouped together on the same line. \\

% \centerline{$n^{2n^3} \quad n! \quad n\log(n^2) \quad 2n^3 \quad 100 \quad n^{50} \quad 50^n \quad \log(n!) \quad \sqrt[3]{n} \quad n^{2n} $} \\ 
% \centerline{$\log^2(n) \quad 3\log(n) \quad n^{-3/2} \quad 3^n \quad 100^{-n} \quad n\log(n) \quad \frac{1}{\log(n)} \quad 2^{n^2} \quad \sqrt[2]{n^2}$} \\

% \section*{Recurrence Relations} 
% \subsection*{Question 5: \hfill[4]}
% Use the Master Theorem to find the closed form of the following recurrence relation: \\
% \centerline{$T(n) = 3T(n/5) + 2n$}

\newpage 
\section*{Order Statistics}
The selection problem is to find the $ith$ order statistic of a list. In class, we studied the deterministic linear-time selection algorithm SELECT (also known as Median of Medians).

\subsection*{Question 3: \hfill[5]}
Show how to use SELECT as a subroutine to make Quicksort run in $O(n\log(n))$ time in the worst case, assuming that all elements are distinct. \\
(* Refer to relevant parts of CLRS book for a description of Quicksort).

% \subsection*{Question 7:}
% In the algorithm SELECT, the input elements are divided into groups of 5. Show that the algorithm works in linear time if the input elements are divided into groups of 7 instead of 5. \\
% (* Page 241 of Introduction to Algorithms, Thomas H. Cormen).

\subsection*{Question 4: \hfill[5]}
The SELECT algorithm divides the list into subgroups of size $c$, where $c$ is typically set to $5$. What effect would there be on the algorithm's worst-case run-time if is chosen to be a) less than $5$, b) more than $5$, and c) if $c$ becomes a function of $n$, i.e., $\log n, \sqrt{n}, n/2, n$?


\subsection*{Question 5: \hfill [5]}
Design a linear time algorithm to find all of the $\log n$ smallest integers, {\color{blue} in sorted order}, using the SELECT problem as a subroutine. For example, if we have a list of $n=2^{100}$ integers, then we are interested in the $100$ smallest integers in the given list.

% \subsection*{Question 6:}
% Suppose you have a list $X$ with elements $x_i$, and another list $W$ with weights $w_i$ for each element in $x_i$ in $X$. 
% Design a linear-time algorithm to find an element $x$ with weight $w$ such that the sum of weights of elements less than x is at most 1/2 the total weights, and the sum of weights of elements greater than x is at most 1/2 the total weights. 

% \subsection*{Question 7:}
% Prove that any comparison-based algorithm that finds the median must perform $\Omega(n)$ comparisons in the worst case.
% \\
\section*{Randomized Algorithms}
RANDOMIZED-SELECT is a selection algorithm where pivots are chosen {\em uniformly} at random, unlike the Median of Medians problem which is a deterministic-linear algorithm.
\subsection*{Question 6: \hfill[5]}
Show an example where RANDOMIZED-SELECT has a worst-case run-time of $O(n^2)$

\subsection*{Question 7: \hfill[5]}
Compute the probability that a randomly chosen pivot splits the array into two parts each of size at least $\frac{3n}{10}$. Use this to show that the expected run-time of RANDOMIZED-SELECT is $\Theta(n)$.

\subsection*{Question 8: \hfill[5]}
Suppose we want RANDOMIZED-SELECT to succeed in finding the $k$-th smallest element \emph{without hitting a worst-case path}. If the probability of a good split is $p$, what is the probability that after $t$ independent pivot choices, we always hit good splits?
  



% \section*{Question 3: Dynamic Programming -- Inventory Management (20 marks)}
% A store sells a product over $n$ days. The demand for day $i$ is $d_i$ units. The store can produce up to $P$ units per day. Producing more than $P$ units on any day incurs a penalty of $c$ per extra unit. Additionally, storing inventory costs $h$ per unit per day.

% \begin{itemize}
%     \item Formulate a DP solution to minimize the \textbf{total cost} over $n$ days.
%     \item Define your state, transitions, and base cases clearly.
%     \item Analyze the time complexity of your algorithm.
% \end{itemize}

% \section*{Question 4: Graphs -- Unusual Shortest Paths (15 marks)}
% Given a directed graph with \textbf{non-negative edge weights}, and two nodes $s$ and $t$, find the shortest path from $s$ to $t$ that uses \textbf{exactly $k$ edges}.

% \begin{itemize}
%     \item Explain why Dijkstra's algorithm doesn't work here.
%     \item Devise an algorithm (preferably DP-based or graph transformation-based) to solve the problem.
%     \item Analyze its correctness and complexity.
% \end{itemize}



% \section*{Question 1: Greedy Algorithm -- Network Coverage (15 marks)}
% You are tasked with placing Wi-Fi routers in a linear corridor consisting of $n$ rooms. Each router can cover a continuous segment of $r$ rooms. Each room $i$ has an associated value $v_i$ representing its importance. 

% \begin{itemize}
%     \item Design a greedy algorithm to place at most $k$ routers to maximize the total value of covered rooms.
%     \item Prove or disprove the optimality of your greedy approach.
%     \item If not optimal, explain with a counterexample and suggest a better method.
% \end{itemize}

% \section*{Question 2: Constrained Huffman Coding (15 marks)}
% You are given a set of symbols with frequencies. One of the symbols, say $x$, must be assigned a binary code of length exactly 3. 

% \begin{itemize}
%     \item Can the standard Huffman algorithm accommodate this constraint? Explain.
%     \item If yes, show how to modify the algorithm accordingly.
%     \item If no, justify why and describe an alternative encoding strategy.
% \end{itemize}

% \section*{Question 3: Dynamic Programming -- Staircase Variants (20 marks)}
% A robot is climbing a staircase with $n$ steps. At each step, it can hop 1, 2, or 3 stairs. Some steps are broken and cannot be stepped on. Let $B$ be the set of broken steps.

% \begin{itemize}
%     \item Formulate a dynamic programming solution to count the number of distinct valid ways to reach the top.
%     \item Analyze the time and space complexity.
% \end{itemize}

% \section*{Question 3: Dynamic Programming -- Grid Path with Turns (20 marks)}
% Write a dynamic programming algorithm to answer the (decision version of) the Partition Problem. Analyze the time complexity. Make observations about the cases where the problem is in P and where it is NP-complete.


% \section*{Question 4: Shortest Path with Fuel Stops (15 marks)}
% You are given a graph where each edge has a distance. A vehicle can travel at most $d$ distance units on a full tank and can refuel at specific nodes only. Design an algorithm to find the shortest path from node $s$ to node $t$ considering fuel constraints.

% \begin{itemize}
%     \item Describe how to model this problem using graph transformation or state expansion.
%     \item Provide an efficient algorithm.
%     \item Analyze the correctness and complexity.
% \end{itemize}

% \section*{Question 5: NP-Completeness -- Set Packing (15 marks)}
% The \textbf{Set Packing problem} is defined as follows: Given a universe $U$ and a collection of subsets $S_1, S_2, \dots, S_m$ of $U$, and an integer $k$, is there a collection of $k$ pairwise disjoint sets?

% \begin{itemize}
%     \item Prove that Set Packing is in NP.
%     \item Reduce from 3-SAT to Set Packing to show NP-completeness.
%     \item Justify each step of your reduction.
% \end{itemize}

% \section*{Question 6: NP-Hardness -- 3-Partition Problem (20 marks)}
% The \textbf{3-Partition problem} is defined as follows: Given a multiset $A$ of $3m$ positive integers and a bound $B$, can $A$ be partitioned into $m$ disjoint sets $A_1, A_2, ..., A_m$ such that the sum of numbers in each set is exactly $B$ and each set contains exactly 3 elements?

% \begin{itemize}
%     \item Define the decision version of the problem.
%     \item Show that 3-Partition is in NP.
%     \item Prove that 3-Partition is strongly NP-complete by reduction from Partition or another appropriate problem.
% \end{itemize}

% \section*{Question 1: Greedy Algorithm -- Activity Selection with Constraints (15 marks)}
% You are given a list of $n$ activities with start and end times. You may select non-overlapping activities to perform. Additionally, you are not allowed to select two consecutive activities (i.e., no back-to-back scheduling).

% \begin{itemize}
%     \item Modify the standard greedy approach to handle this new constraint.
%     \item Prove whether the modified greedy solution is optimal.
%     \item Provide an example input and show the selected activities.
% \end{itemize}

% \section*{Question 2: Huffman Tree with Prefix Constraint (15 marks)}
% Construct a Huffman tree for a set of characters with frequencies, under the constraint that one symbol must be assigned a binary code that starts with `01`.

% \begin{itemize}
%     \item Discuss how this prefix constraint alters the tree construction process.
%     \item Propose an algorithm to modify Huffman coding accordingly.
%     \item Justify if the result is optimal or not.
% \end{itemize}

% \section*{Question 3: Dynamic Programming -- Grid Path with Turns (20 marks)}
% Given an $n \times m$ grid, find the minimum-cost path from the top-left to the bottom-right cell. You may move only right or down. Each move has a cost, and changing direction (e.g., from right to down) adds an extra cost $c$.

% \begin{itemize}
%     \item Formulate a DP algorithm to solve this.
%     \item Clearly define the states and transitions.
%     \item Analyze the time complexity.
% \end{itemize}

% \section*{Question 4: Shortest Path -- Max Bandwidth Path (15 marks)}
% In a graph where each edge has a bandwidth value, design an algorithm to find a path from source $s$ to destination $t$ such that the \textbf{minimum bandwidth} among all edges on the path is maximized (known as the widest path problem).

% \begin{itemize}
%     \item Present an efficient algorithm.
%     \item Discuss the data structures used.
%     \item Analyze the time complexity.
% \end{itemize}

% \section*{Question 5: NP-Completeness -- Hamiltonian Path (15 marks)}
% A \textbf{Hamiltonian path} in a graph is a path that visits each vertex exactly once. 

% \begin{itemize}
%     \item Define the decision version of the Hamiltonian Path problem.
%     \item Show it is in NP.
%     \item Reduce from 3-SAT or another NP-complete problem to prove NP-completeness.
% \end{itemize}

% \section*{Question 6: NP-Hardness -- Job Scheduling on Two Machines (20 marks)}
% You are given a list of $n$ jobs, each with a processing time. Can you assign the jobs to two machines such that the maximum load on any machine is minimized? 

% \begin{itemize}
%     \item Formulate the decision problem.
%     \item Show that this problem is NP-hard.
%     \item Provide a reduction from Partition to support your claim.
% \end{itemize}

\end{document}
