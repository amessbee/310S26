% -*- program: xelatex -*-

\newif\ifanswers
%\answerstrue

% Dr Driver's standard reading quiz
% Source: https://gist.github.com/danieldriver/90a73c4d3c72dd837e39#file-quiz-tex 
\documentclass[%
% 11pt,
% answers,
addpoints]{exam}

\pagestyle{head}

\usepackage{amssymb,amsmath,amsfonts,amsthm,graphicx}

\usepackage[top=0.3in, bottom=0.75in, left=0.5in, right=0.5in]{geometry}
\usepackage{amsmath,amssymb}
\usepackage{tikz}

\DeclareGraphicsExtensions{.pdf}
%\centerline {\includegraphics[width=3in]{PICTURE}}



\newtheoremstyle{problem}{\topsep}{\topsep}%%% space between body and thm
{}                      %%% Thm body font
{}                              %%% Indent amount (empty = no indent)
{\bfseries}            %%% Thm head font
{}                    %%% Punctuation after thm head
{ }                           %%% Space after thm head
{\thmnumber{#2}\thmnote{ \bfseries (#3)}}%%% Thm head spec
\theoremstyle{problem}
\newtheorem{p}{}

\firstpageheader{% left
	CS \liningnums{310}: Algorithms --- Divide And Conquer\\
	Mudassir Shabbir, February 04, 2026
}{% center - blank
}{% right
	Full name1:\enspace\makebox[2in]{\hrulefill}\\
	Full name2:\enspace\makebox[2in]{\hrulefill}\\
}
\runningheader{}{}{}
% For double-sided quizzes
% \pagestyle{headandfoot}
% \footer{}{\thepage}{}

% Typography and layout
\usepackage{fontspec,realscripts}
\defaultfontfeatures{Ligatures=TeX}
%\setmainfont{Meta Serif Pro}
%\setsansfont{Meta Pro}
\frenchspacing
% - print solutions in sans serif
\unframedsolutions
\SolutionEmphasis{\sffamily}
\renewcommand{\solutiontitle}{}
% - box points & center in the right margin w/ custom setup@point@toks
\boxedpoints
\pointsinrightmargin
\marginbonuspointname{\textsc{up}}
\makeatletter% rewrite setup@point@toks assuming right margins
\def\clap#1{\hbox to 0pt{\hss#1\hss}}% define \clap as per https://www.tug.org/TUGboat/tb22-4/tb72perlS.pdf
\def\setup@point@toks{%
	\point@toks={%
		\rlap{\hskip-\@totalleftmargin
			\hskip\textwidth
			\hskip\@rightmargin
			\hskip-\rightpointsmargin
			\clap{\padded@point@block}% change \llap to \clap
		}%
		\global \point@toks={}%
	}%
}% end setup@point@toks
\setlength{\rightpointsmargin}{.5in}% assuming the default 1" margins
\makeatother
% - adjust the top and bottom margins
\extraheadheight{.25in}
\extrafootheight{-.5in}
\setlength{\marginparwidth}{1.5in}
% NB: remember to use \newpage after the last question
\usepackage{xcolor}
\usepackage{pagecolor}

%\pagecolor{black}
%\color{white}

\usepackage[document]{ragged2e}
\begin{document}
	
	\pagestyle{empty}
	\noindent
\begin{tabular*}{\textwidth}{@{\extracolsep{\fill}} l r}
\textbf{CS 310: Algorithms -- Recurrences, Divide and Conquer} 
& \emph{Mudassir Shabbir, February 04, 2026}
\end{tabular*}

\vspace{0.4cm}

\noindent
\begin{tabular*}{\textwidth}{@{\extracolsep{\fill}} l l}
\textbf{Student ID 1 \& 2:}{\em (Do this in pairs)} \hrulefill 
& 
\end{tabular*}

\vspace{0.3cm}
\hrule


	\thispagestyle{myheadings}
	\pagenumbering{gobble}
	\begin{p}
		\textbf{Solve the following recurrence relations:}
\begin{enumerate}
    \item $T(n) = 2T(n/3) + 1$
    \vspace{1.2cm}
    \item $T(n) = 5T(n/4) + n$
    \vspace{1.2cm}
    \item $T(n) = 2T(n - 1) + 1$
    \vspace{1.2cm}
    \item $T(n) = 9T(n/3) + n^2$
    \vspace{1.2cm}
\end{enumerate}
		\hfill \end{p}
	%\vfill
	\ifanswers
	{\bf SOL:}  
	\vfill
	\fi
	
	\begin{p} 
        \textbf{Majority Element Problem}

An array $A[1 \ldots n]$ is said to have a \emph{majority element} if more than half of its entries are the same.
Given an array, the task is to design an efficient algorithm to determine whether the array has a majority element,
and, if so, to find that element.
The elements of the array are not necessarily from an ordered domain (e.g., integers), so comparisons of the form
``$A[i] > A[j]$'' are not allowed. You may only answer questions of the form ``$A[i] = A[j]$'' in constant time.
(Think of the array elements as GIF files, for instance.)

\vspace{0.2cm}

\textbf{Part 1:} Show how to solve this problem in $O(n \log n)$ time.\footnote{\emph{Hint:} Split the array $A$ into two arrays $A_1$ and $A_2$ of half the size. Does knowing the majority elements
of $A_1$ and $A_2$ help you determine the majority element of $A$? If so, use a divide-and-conquer approach.}

\newpage
\vspace*{50pt}
\textbf{Part 2:} Can you give a linear-time algorithm?\footnote{\emph{Hint:} Consider the following divide-and-conquer strategy:
\begin{itemize}
    \item Pair up the elements of $A$ arbitrarily to form $n/2$ pairs.
    \item For each pair: {\em If the two elements are different, discard both.
If they are the same, keep exactly one of them.}
\end{itemize}

Show that after this procedure there are at most $n/2$ elements left, and that the remaining elements have a majority
element if and only if the original array $A$ does. Prove time complexity.}

		\hfill \end{p}
	%\vfill
	\ifanswers 
	{\bf SOL:} $ ( P \vee Q ) \wedge (\neg R) $
	\vfill
	\fi
		
	
\end{document}
